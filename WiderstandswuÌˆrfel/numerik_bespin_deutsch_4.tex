\documentclass[10pt,a4paper]{article}
\usepackage[utf8]{inputenc}
\usepackage[german]{babel}
\usepackage{amsmath}
\usepackage{amsfonts}
\usepackage{amssymb}
\usepackage{siunitx}
\usepackage[left=2cm,right=2cm,top=2cm,bottom=2cm]{geometry}
\usepackage{wrapfig}
\usepackage{graphicx}
\usepackage[outdir=./]{epstopdf}
\usepackage{caption}
\usepackage[colorlinks]{hyperref}
\usepackage{pdflscape}

\author{Christian Bespin \and Christopher Deutsch}
\title{Übungsblatt 4: Numerische Methoden der Physik}
\begin{document}
\maketitle

\setcounter{section}{1}

\section{Widerstandswürfel}

\subsection{Physikalischer Hintergrund}

%\begin{wrapfigure}[14]{R}[1pt]{0.47\textwidth}
%\centering
%\includegraphics[width=0.45\textwidth]{./figures/GRAFIK}
%\caption{CAPTION}
%\label{fig:CAPTION}
%\end{wrapfigure}


\subsubsection{Oktaeder}
<<<<<<< HEAD
=======



>>>>>>> ad5151400e2a581f1436154324e116114c45128e

\subsection{Implementierung}

\subsubsection{Nullstellenberechnung mit Bisektion}

\subsubsection{Numerische Integration mit rekursivem Simpson-Verfahren}


\subsubsection{Abbruchbedingungen}

\subsubsection{Genaugkeit}

\subsection{Physikalische Ergebnisse}
\begin{thebibliography}{9}

\bibitem{lyness}
 Lyness, J. N.,
 \emph{Notes on the Adaptive Simpson Quadrature Routine},
Journal of the ACM
Volume 16 Issue 3, (1969) 
Seiten 483-495 

\end{thebibliography}
\begin{landscape}
\thispagestyle{empty}
\appendix
\section{Anhang}
\subsection{Oktaeder}
\begin{align}
\begin{pmatrix}
	0 & 0 & -R_4 & 0 & 0 & -R_{12} & 0 & R_4 + R_{12} \\
	R_1 + R_2 + R_5 & -R_2 & 0 & -R_5 & 0 & 0 &-R_5 & 0 \\
	-R_2 & R_2 + R_3 + R_6 & -R_3 & 0 & -R_6 & 0 & -R_6 & 0 \\
	0 & -R_3 & R_3 + R_4 + R_7 & 0 & 0 & -R_7 & -R_7 & -R_4 \\
	-R_5 & 0 & 0 & R_5 + R_9 + R_{10} & -R_{10} & 0 & R_5 & 0 \\
	0 & -R_6 & 0 & -R_{10} & R_6 + R_{10} + R_{11} & -R_{11} & R_6 & 0 \\
	0 & 0 & -R_7 & 0 & -R_{11} & R_7 + R_{11} + R_{12} & R_7 & -R_{12} \\
	-R_5 & -R_6 & -R7 & R_5 & R_6 & R_7 & R_5 + R_6 + R_7 + R_8 & 0
\end{pmatrix}
\end{align}
\subsection{Würfel}
\begin{align}
\begin{pmatrix}
R_1+R_2+R_3+R_4 &  -R_4  &  0  &  -R_3  &  -R_2  &  0  \\ 
-R_4 & R_4+R_6+R_7+R_{10} & -R_{10} & -R_7 & -R_6 & -R_6 \\ 
 0  & -R_{10} & (R_9+R_{10}+R_{11}+R_{12}) & -R_{11} & -R_9 & -(R_9+R_{12}) \\ 
-R_3 & -R_7 & -R_11 & (R_3+R_7+R_8+R_{11}) & 0 & 0 \\ 
-R_2 & -R_6 & -R_9 & 0 & (R_2+R_5+R_6+R_9) & (R_6+R_9) \\ 
 0  & -R_6 & -(R_9+R_{12}) &  0  & R_6+R_9 & R_6+R_9+R_{12}
\end{pmatrix}
\end{align}
\begin{pmatrix}
	0 & 0 & -R_4 & 0 & 0 & -R_{12} & 0 & R_4 + R_{12} \\
	R_1 + R_2 + R_5 & -R_2 & 0 & -R_5 & 0 & 0 &-R_5 & 0 \\
	-R_2 & R_2 + R_3 + R_6 & -R_3 & 0 & -R_6 & 0 & -R_6 & 0 \\
	0 & -R_3 & R_3 + R_4 + R_7 & 0 & 0 & -R_7 & -R_7 & -R_4 \\
	-R_5 & 0 & 0 & R_5 + R_9 + R_{10} & -R_{10} & 0 & R_5 & 0 \\
	0 & -R_6 & 0 & -R_{10} & R_6 + R_{10} + R_{11} & -R_{11} & R_6 & 0 \\
	0 & 0 & -R_7 & 0 & -R_{11} & R_7 + R_{11} + R_{12} & R_7 & -R_{12} \\
	-R_5 & -R_6 & -R7 & R_5 & R_6 & R_7 & R_5 + R_6 + R_7 + R_8 & 0
\end{pmatrix}
\end{landscape}


\end{document}
