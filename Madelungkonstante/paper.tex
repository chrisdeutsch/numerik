\documentclass[10pt,a4paper]{article}
\usepackage[utf8]{inputenc}
\usepackage[german]{babel}
\usepackage{amsmath}
\usepackage{amsfonts}
\usepackage{amssymb}
\usepackage[left=2cm,right=2cm,top=2cm,bottom=2cm]{geometry}

\usepackage{listings}

\author{Christian Bespin \and Christopher Deutsch}
\title{Übungsblatt 1: Numerische Methoden der Physik}
\begin{document}
\maketitle

\section{Madelung-Konstante des NaCl-Kristalls}
\subsection{Physikalischer Hintergrund}
In einem Kristallgitter sind die Ionen durch elektrostatische Wechselwirkung untereinander an den Kristall gebunden.
Uns interessiert die durchschnittliche potentielle Energie eines dieser Ionen (wird mit dem Index $i$ bezeichnet).
Diese entsteht in erster Näherung aus der Überlagerung der Coulomb-Potentiale der Punktladungen im Kristallgefüge
(wird mit dem Index $j$ bezeichnet). Die Ladungszahl des jeweiligen Ions sei $z$ und $r_{ij}$ ist der Abstand des
betrachteten Ionenpaars. Wir summieren für jedes Ion des Kristalls (mit Ausnahme des $i$-Ions, da dessen Potential
auf sich selbst keinen Einfluss hat) den Beitrag zum Potential:

\begin{align}
\label{EPot}
E_G = \sum_{j\atop i \neq j} \frac{1}{4 \pi \epsilon_0}  \frac{z_i e \cdot z_j e}{r_{ij}}
\end{align}
Erweitern der Gleichung (\ref{EPot}) mit dem Gitterabstand $a$ und Zusammenfassen des Faktors in die durchschnittliche
Energie eines Ionenpaars $E_P$ mit Abstand $a$ ergibt:

\begin{align}
E_G = \frac{1}{4 \pi \epsilon_0} \frac{e^2}{a} \sum_{j\atop i \neq j} \frac{z_i z_j}{r_{ij}/a} = E_P \sum_{j\atop i \neq j} \frac{z_i z_j}{r_{ij}/a}
\end{align}
Schließlich wird die Geometrie des Kristalls in der Madelung-Konstante zusammengefasst:

\begin{align}
\alpha = \frac{E_G}{E_P} = \sum_{j\atop i \neq j} \frac{z_i z_j}{r_{ij}/a}
\end{align}
TODO: noch was zum normalisierten Abstand sagen; es macht keinen Sinn das Zentralion mit i zu indizieren, da darüber niemals summiert wird

\subsection{Algorithmus im 3-dimensionalen Fall}
\subsubsection{Struktur von Natriumchlorid}

Die nächsten Nachbarn eines jeden Ions sind sechs Ionen entgegengesetzter Ladung. Diese Ionen befinden sich an den Eckpunkten eines regelmäßigen Oktaeders. Dadurch kann das Vorzeichen der Ladung an der Gitterstelle $\mathbf{r} = \left( x,y,z \right)$ durch:

\begin{align}
\frac{z}{|z|}(x,y,z) = \pm \left( -1 \right)^{x+y+z}
\end{align}
berechnet werden. Dabei wird das Vorzeichen so gewählt, dass dies mit dem Ion im Zentrum des Würfels übereinstimmt (Na: $+$; Cl: $-$). Diese Tatsache äußert sich auch darin, dass das Vorzeichen der Madelung-Konstante vom betrachteten Ion abhängt $\alpha_{Na} = - \alpha_{Cl}$.
\subsubsection{erster Versuch naive Methode}

\begin{itemize}
\item Vorgehen analog zur mathematischen Beschreibung
\item Symmetrie wird ignoriert (ineffizient)
\item der betrachtete Würfel bleibt nicht neutral
\item naiver Versuch kurz beschreiben und sagen, dass der mies ist
\item wie ein chef
\end{itemize}

\subsubsection{Beschreibung}

\begin{itemize}
\item ab hier wirds dann serious und unseren abgegebenen code betreffend
\item Gewichtung der einzelnen Ladungen nach der Methode von Evjens\cite{Evjen}
\item es wird über Elementarzellen summiert die wie üblich gewichtet werden. in der summe über einen ganzen würfel bleibt dann die gewichtung wie wir sie im programm implementiert haben. (Evjens PDF seite 7)
\end{itemize}

\subsubsection{Konvergenz}

\begin{itemize}
\item erkläre physikalischen Hintergrund für die bessere Konvergenz der Evjen Methode
\item das Potential einer neutralen Elementarzelle fällt schneller ab als das einer geladenen
\end{itemize}

\subsection{Algorithmus im 2-dimensionalen Fall}
\subsubsection{Beschreibung}

In diesem Fall wird angenommen, dass $j_3 = i_3$.
Hier brauchen wir vermutlich eine neue Skalierungsmethode. In der Tat, Ecken $\frac{1}{4}$ Kanten $\frac{1}{2}$.

\subsubsection{Konvergenz}

Wie gut konvergiert der 2-dimensionale Algorithmus.

\subsection{Fazit}

Verwendeter Literaturwert $\alpha_{NaCl} = 1.7475646946331822$ \cite{Sakamoto} .


\begin{thebibliography}{9}

\bibitem{Evjen}
Evjen, H. M.
\emph{On the Stability of Certain Heteropolar Crystals},
Physical Review Letters \textbf{39},
675-687 (1932)

\bibitem{Sakamoto}
Sakamoto, Y.
\emph{Madelung Constants of Simple Crystals Expressed in Terms of Born's Basic Potentials of 15 Figures},
The Journal of Chemical Physics \textbf{28},
164 (1958)

\end{thebibliography}

\end{document}
