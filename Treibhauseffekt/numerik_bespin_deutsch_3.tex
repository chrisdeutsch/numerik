\documentclass[10pt,a4paper]{article}
\usepackage[utf8]{inputenc}
\usepackage[german]{babel}
\usepackage{amsmath}
\usepackage{amsfonts}
\usepackage{amssymb}
\usepackage{siunitx}
\usepackage{multirow}
\usepackage[left=2cm,right=2cm,top=2cm,bottom=2cm]{geometry}
\usepackage{wrapfig}
\usepackage{graphicx}
\usepackage{caption}
\usepackage[colorlinks]{hyperref}


\author{Christian Bespin \and Christopher Deutsch}
\title{Übungsblatt 3: Numerische Methoden der Physik}
\begin{document}
\maketitle

\setcounter{section}{1}

\section{Treibhauseffekt}

\subsection{Physikalischer Hintergrund}

Wir verwenden zur Erklärung des Treibhauseffektes ein einfaches Modell grauer und schwarzer Körper. Die Erde entspricht dabei einem schwarzen Körper, das heißt Emissionsgrad ist gleich $1$, mit der Temperatur $T_E$. Die Sonne wird als grauer Körper mit konstanter Emissivität $\epsilon_S$ und ebenfalls konstanter Temperatur $T_S$. Im Gegensatz zum schwarzen Körper, absorbiert der graue Körper nicht die gesamte auf ihn treffende Strahlung und emittiert entsprechend auch nicht die maximal mögliche Schwarzkörperstrahlung. Es gilt daher, dass $\epsilon_{S}<1$. Im verwendeten Modell hat die Atmosphäre die Emissivität $\epsilon$ und die Temperatur $\tau$. Die Atmosphäre absorbiert den Anteil $epsilon(T_S)$ der von der Sonne emittierte Srahlung, lässt entsprechend den Anteil $1-epsilon(T_S)$ dieser Strahlung zur Erde durch und emittiert selber eine Strahlung mit Anteil $\epsilon(\tau)}$. Letzteres , sowie die zur Berechnung benötigten Formeln sind unter $(1)$-$(3)$ auf dem Aufgabenzettel zu finden.

\subsection{Mathematischer Hintergrund}

\subsection{Implementierung}

\subsubsection{Struktur des Programmes}

\subsubsection{Abbruchbedingungen}

\subsubsection{Genaugkeit}

\subsection{Physikalische Ergebnisse}
\label{ssec:physikalischeergebnisse}

\begin{thebibliography}{9}

\bibitem{abramowitzstegun}
Abramowitz, M. \& Stegun, I. A.
\emph{Handbook of Mathematical Functions},
Dover Books (1965)

\bibitem{healdmarion}
Heald, M. \& Marion, J.
\emph{Classical Electromagnetic Radiation},
Brooks Cole (1994)

\bibitem{kazimierczuk}
Kazimierczuk, M.
\emph{High-Frequency Magnetic Components},
Wiley (2009)

\bibitem{crchandbook}
David R. Lide (ed),
\emph{CRC Handbook of Chemistry and Physics},
84th Edition. CRC Press. Boca Raton, Florida, 2003;
Section 12, Properties of Solids; Electrical Resistivity of Pure Metals;
Section 4, The Elements: Magnetic Susceptibility

\end{thebibliography}

\end{document}
