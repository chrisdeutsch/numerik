\documentclass[10pt,a4paper]{article}
\usepackage[utf8]{inputenc}
\usepackage[german]{babel}
\usepackage{amsmath}
\usepackage{amsfonts}
\usepackage{amssymb}
\usepackage{multirow}
\usepackage[left=2cm,right=2cm,top=2cm,bottom=2cm]{geometry}
\usepackage{wrapfig}
\usepackage{graphicx}
\usepackage{caption}
\usepackage[colorlinks]{hyperref}


\author{Christian Bespin \and Christopher Deutsch}
\title{Übungsblatt 2: Numerische Methoden der Physik}
\begin{document}
\maketitle

\setcounter{section}{1}

\section{Skin-Effekt}

\subsection{Physikalischer Hintergrund}

In einem zylinderförmigen elektrischen Leiter, der hinreichend lang ist und in dem ein Wechselstrom $I(t)=I_0\cdot\text{e}^{\text{i}\omega t}$ fließt, ist die Stromdichte nicht über den ganzen Leiterquerschnitt konstant. Das im Leiter schwingende elektrische Feld genügt der Gleichung\cite{landau}:
\begin{align}
	\Delta\mathbf{E}=\frac{4\pi\sigma}{c^2}\cdot \frac{\partial \mathbf{E}}{\partial t}
	\label{eq:maxwellgleichung}
\end{align}
Legt man die z-Achse parallel zum Leiter, so reduziert sich die Gleichung auf eine Komponente von E. Da für die Stromdichte $j=\sigma E$ gilt, erhält man für $j$ (in Zylinderkoordinaten) die Besselsche Differentialgeichung
\begin{align}
	\frac{\mathrm{d}^2 j}{\mathrm{d}\rho^2} + \frac{1}{\rho}\frac{\mathrm{d}j}{\mathrm{d}\rho} - k^2\cdot j=0
	\label{eq:besseldgl} 
\end{align}
mit $k^2=\mathrm{i}\omega\mu\sigma$. Als Lösung\cite{kazimierczuk} erhält man eine Bessel-Funktion erster Art von nullter Ordnung, die man mit den Kelvin-Funktionen in Real- und Imaginärteil zerlegen kann:
\begin{align}
	j(\rho)=A J_0(k\rho)=A(\mathrm{ber}_0(k\rho)+\mathrm{i bei}_0(k\rho))
\end{align}
$A$ kann aus dem Strom, der im Leiter fließt, zu 
\begin{align}
A=\frac{I_0 k}{2\pi\rho_0}\frac{1}{\mathrm{bei}'_0(k\rho_0)-\mathrm{i ber}'_0(k\rho_0)}
\end{align}bestimmt werden\cite{kazimierczuk}, woraus direkt
\begin{align}
	j(\rho)=A J_0(\kappa\rho)=\frac{I_0 \kappa}{2\pi\rho_0}\frac{\mathrm{ber}_0(\kappa\rho)+\mathrm{i bei}_0(\kappa\rho)}{\mathrm{bei}'_0(\kappa\rho_0)-\mathrm{i ber}'_0(\kappa\rho_0)}
\end{align}
mit $\kappa=2\frac{\sqrt{\pi\mu\sigma\omega}}{c^2}$folgt. Die Stromdichte ist zum Rand des Leiters hin größer, wie man, mit Kenntnis über den Verlauf der Besselfunktion, herausfinden kann. Dieser Effekt wird als \emph{Skin-Effekt} bezeichnet, der hier nach erfolgreicher Implementierung der Kelvin-Funktionen an einigen Werten beispielhaft numerisch überprüft werden soll. Der Ausduck für $j$ beinhaltet komplexe Zahlen, da $j$ proportional zu $E$ ist. Da der Leiter sowohl eine Kapazität, als auch eine Induktivität hat, treten also komplexe Wechselstromwiderstände auf, die zu komplexen Komponenten in $E$ und damit $j$ führen. Sie lassen sich als \emph{Blindwerte} vorstellen, analog zu Blindwiderstand oder Blindleistung.
\subsection{Mathematischer Hintergrund}

\subsubsection{Bessel-Funktionen erster Art}

Man kann die Bessel-Funktionen erster Art durch ihre Potenzreihenentwicklung im Ursprung definieren:
\begin{align}
	J_\nu(z) = \sum^{\infty}_{m=0} \frac{\left( -1 \right)^m}{m! \, \Gamma(m + \nu + 1)} \left(\frac{z}{2}\right)^{2m+\nu}
\end{align}
Dabei ist $\nu \in \mathbb{R}$. Besonders interessant ist die Besselfunktion 0-ter Ordnung $\nu = 0$.
Mit der Identität $\Gamma(n+1) = n!$ für $n \in \mathbb{N}$ ergibt sich:
\begin{align}
	\label{eq:bessel0}
	J_0(z) = \sum^{\infty}_{m=0} \frac{\left( -1 \right)^m}{\left( m! \right)^2} \left( \frac{z}{2} \right)^{2m}
\end{align}

\subsubsection{Kelvin-Funktionen}

Die Kelvin-Funktionen $\mathrm{ber}_\nu(x)$ und $\mathrm{bei}_\nu(x)$ sind definiert als der Real-
beziehungsweise Imaginärteil von $J_\nu(x \sqrt{-i} )$. Mit \ref{eq:bessel0}
folgt für den Fall $\nu = 0$ umgehend die Reihendarstellung der Kelvinfunktionen:
\begin{align}
	\mathrm{ber}_0(x) &= \operatorname{Re}\left(J_0(x \sqrt{-i} )\right) = 1 + \sum^{\infty}_{k=1} \frac{\left( -1 \right)^k}{\left( \left(2k\right)! \right)^2} \left( \frac{x}{2} \right)^{4k} \label{eq:berpotenzreihe}\\
	\mathrm{bei}_0(x) &= \operatorname{Im}\left(J_0(x \sqrt{-i} )\right) = \sum^{\infty}_{k=0} \frac{\left( -1 \right)^k}{\left( \left(2k+1\right)! \right)^2} \left( \frac{x}{2} \right)^{4k+2}\label{eq:beipotenzreihe}
\end{align}

Für die Ableitung muss \ref{eq:berpotenzreihe} und \ref{eq:beipotenzreihe} nur gliedweise abgeleitet werden.
Dann ergibt sich:
\begin{align}
	\mathrm{ber}'_0(x) &= \sum^{\infty}_{k=1} \frac{\left( -1 \right)^k}{\left( 2k-1 \right)! \, \left( 2k \right)!} \left( \frac{x}{2} \right)^{4k-1}\\
	\mathrm{bei}'_0(x) &= \sum^{\infty}_{k=0} \frac{\left( -1 \right)^k}{\left( 2k \right)! \, \left( 2k+1 \right)!} \left( \frac{x}{2} \right)^{4k+1}
\end{align}

In asymptotischer Näherung\footnote{Hier was zu der Näherung der asymptotischen Näherung und Quelle} ergibt sich mit der Definition $\alpha = \frac{x}{\sqrt{2}}-\frac{\pi}{8}$ und $C(x) = e^{\frac{x}{\sqrt{2}}}/\sqrt{2 \pi x} $:
\begin{align}
	\mathrm{ber}_0(x) &\approx C(x) \left[f_0(x) \cos(\alpha) + g_0(x) \sin(\alpha) \right]\label{eq:berpotenzreiheasym}\\
	\mathrm{bei}_0(x) &\approx C(x) \left[f_0(x) \sin(\alpha) - g_0(x) \cos(\alpha) \right]\label{eq:beipotenzreiheasym}
\end{align}

Mit den Hilfsfunktionen:
\begin{align}
  f_0(x) = 1 + \sum^{\infty}_{k=1} \frac{\cos(k \pi / k)}{k! \, (8k)^k} \prod^{k}_{l=1}(2l - 1)^2\\
  g_0(x) = \sum^{\infty}_{k=1} \frac{\sin(k \pi / k)}{k! \, (8k)^k} \prod^{k}_{l=1}(2l - 1)^2
\end{align}

Auch dieses leiten wir wieder ab:
\begin{align}
	\mathrm{ber}'_0(x) &\approx C(x) \left( \frac{1}{\sqrt{2}} - \frac{1}{2x} \right) \left[f_0(x) \cos(\alpha) + g_0(x) \sin(\alpha) \right] \notag\\ & + C(x) \left[ f'_0(x) \cos(\alpha) + g'_0(x) \sin(\alpha) - f_0(x) \sin(\alpha)/\sqrt{2} + g_0(x) \cos(\alpha)/\sqrt{2} \right]\label{eq:dberpotenzreiheasym}\\
	\mathrm{ber}'_0(x) &\approx C(x) \left( \frac{1}{\sqrt{2}} - \frac{1}{2x} \right) \left[f_0(x) \sin(\alpha) - g_0(x) \cos(\alpha) \right] \notag\\ & + C(x) \left[ f'_0(x) \sin(\alpha) - g'_0(x) \cos(\alpha) + f_0(x) \cos(\alpha)/\sqrt{2} + g_0(x) \sin(\alpha)/\sqrt{2} \right] \label{eq:dbeipotenzreiheasym}
\end{align}

Und den Ableitungen der Hilfsfunktionen:
\begin{align}
  f'_0(x) = 1 - \sum^{\infty}_{k=1} \frac{\cos(k \pi / k)}{(k-1)! \, 8^k} \frac{1}{x^{k+1}} \prod^{k}_{l=1}(2l - 1)^2\\
  g'_0(x) = -\sum^{\infty}_{k=1} \frac{\sin(k \pi / k)}{(k-1)! \, 8^k} \frac{1}{x^{k+1}} \prod^{k}_{l=1}(2l - 1)^2
\end{align}

\subsection{Implementierung}

Sei $S_k$ der $k$-te Summand der Reihendarstellung von \emph{ber}. Wir berechnen $S_k$ aus $S_{k-1}$ mit
\begin{align}
	S_k=-c\cdot S_{k-1}
\end{align} mit $c=konstant$.
Analog gilt:\newline
\begin{tabular}{c|c|c|c}
\centering
\rule[1ex]{0pt}{2.5ex} c für $\mathrm{ber}$ & c für $\mathrm{bei}$ & c für $\mathrm{ber}'$ & c für $\mathrm{bei}'$ \\ 
\hline 
\rule[2ex]{0pt}{2.5ex} $\dfrac{1}{(2k-1)^2\cdot(2k)^2}\dfrac{x^4}{16}$ & $\dfrac{1}{(2k+1)^2\cdot(2k)^2}\dfrac{x^4}{16}$ & $\dfrac{1}{(2k-2)(2k-1)^2\cdot 2k}\dfrac{x^4}{16}$ & $\dfrac{1}{(2k+1)(2k-1) \cdot (2k)^2}\dfrac{x^4}{16}$ \\ 
\end{tabular}
\begin{thebibliography}{9}

\bibitem{landau}
Landau, L.; Lifshitz, E.
\emph{Electrodynamics Of Contiunous Media},
Pergamon Press (1984)

\bibitem{kazimierczuk}
Kazimierczuk, M.
\emph{High-Frequency Magnetic Components},
Wiley (2009)

\end{thebibliography}

\end{document}
