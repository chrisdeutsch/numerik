\documentclass[10pt,a4paper]{article}
\usepackage[utf8]{inputenc}
\usepackage[german]{babel}
\usepackage{amsmath}
\usepackage{amsfonts}
\usepackage{amssymb}
\usepackage{multirow}
\usepackage[left=2cm,right=2cm,top=2cm,bottom=2cm]{geometry}
\usepackage{wrapfig}
\usepackage{graphicx}
\usepackage{caption}
\usepackage[colorlinks]{hyperref}


\author{Christian Bespin \and Christopher Deutsch}
\title{Übungsblatt 2: Numerische Methoden der Physik}
\begin{document}
\maketitle

\setcounter{section}{1}

\section{Skin-Effekt}

\subsection{Physikalischer Hintergrund}

\subsection{Mathematischer Hintergrund}

\subsubsection{Bessel-Funktionen erster Art}

Man kann die Bessel-Funktionen erster Art durch ihre Potenzreihenentwicklung im Ursprung definieren:
\begin{align}
	J_\nu(z) = \sum^{\infty}_{m=0} \frac{\left( -1 \right)^m}{m! \, \Gamma(m + \nu + 1)} \left(\frac{z}{2}\right)^{2m+\nu}
\end{align}
Dabei ist $\nu \in \mathbb{R}$. Besonders interessant ist die Besselfunktion 0-ter Ordnung $\nu = 0$.
Mit der Identität $\Gamma(n+1) = n!$ für $n \in \mathbb{N}$ ergibt sich:
\begin{align}
	\label{eq:bessel0}
	J_0(z) = \sum^{\infty}_{m=0} \frac{\left( -1 \right)^m}{\left( m! \right)^2} \left( \frac{z}{2} \right)^{2m}
\end{align}


\subsubsection{Kelvin-Funktionen}

Die Kelvin-Funktionen $\mathrm{ber}_\nu(x)$ und $\mathrm{bei}_\nu(x)$ sind definiert als der Real-
beziehungsweise Imaginärteil von $J_\nu(x \sqrt{-i} )$. Mit \ref{eq:bessel0}
folgt für den Fall $\nu = 0$ umgehend die Reihendarstellung der Kelvinfunktionen:
\begin{align}
	\mathrm{ber}_0(x) &= \operatorname{Re}\left(J_0(x \sqrt{-i} )\right) = 1 + \sum^{\infty}_{k=1} \frac{\left( -1 \right)^k}{\left( \left(2k\right)! \right)^2} \left( \frac{x}{2} \right)^{4k}\\
	\mathrm{bei}_0(x) &= \operatorname{Im}\left(J_0(x \sqrt{-i} )\right) = \sum^{\infty}_{k=0} \frac{\left( -1 \right)^k}{\left( \left(2k+1\right)! \right)^2} \left( \frac{x}{2} \right)^{4k+2}
\end{align}

In asymptotischer Näherung\footnote{Hier was zu der Näherung der asymptotischen Näherung und Quelle} ergibt sich:
\begin{align}
	\mathrm{ber}_0(x) &\approx \frac{e^{\frac{x}{\sqrt{2}}}}{\sqrt{2 \pi x}} \cos\left(\frac{x}{\sqrt{2}} - \frac{\pi}{8}\right)\\
	\mathrm{bei}_0(x) &\approx \frac{e^{\frac{x}{\sqrt{2}}}}{\sqrt{2 \pi x}} \sin\left(\frac{x}{\sqrt{2}} - \frac{\pi}{8}\right)
\end{align}

Einfaches gliedweises Ableiten der Potenzreihe liefert:
\begin{align}
	\mathrm{ber}'_0(x) &= \sum^{\infty}_{k=1} \frac{\left( -1 \right)^k}{\left( 2k-1 \right)! \, \left( 2k \right)} \left( \frac{x}{2} \right)^{4k-1}\\
	\mathrm{bei}'_0(x) &= \sum^{\infty}_{k=0} \frac{\left( -1 \right)^k}{\left( 2k \right)! \, \left( 2k+1 \right)} \left( \frac{x}{2} \right)^{4k+1}
\end{align}

\subsection{Implementierung}



\end{document}
