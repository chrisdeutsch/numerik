\documentclass[10pt,a4paper]{article}
\usepackage[utf8]{inputenc}
\usepackage[german]{babel}
\usepackage{amsmath}
\usepackage{amsfonts}
\usepackage{amssymb}
\usepackage{multirow}
\usepackage[left=2cm,right=2cm,top=2cm,bottom=2cm]{geometry}
\usepackage{wrapfig}
\usepackage{graphicx}
\usepackage{caption}
\usepackage[colorlinks]{hyperref}


\author{Christian Bespin \and Christopher Deutsch}
\title{Übungsblatt 2: Numerische Methoden der Physik}
\begin{document}
\maketitle

\setcounter{section}{1}

\section{Skin-Effekt}

\subsection{Physikalischer Hintergrund}

In einem zylinderförmigen elektrischen Leiter, der hinreichend lang ist und in dem ein Wechselstrom $I(t)=I_0 e^{i \omega t}$ fließt, ist die Stromdichte nicht über den ganzen Leiterquerschnitt konstant. Das im Leiter schwingende elektrische Feld genügt mit $\dot D \approx 0$ der elektromagnetischen Wellengleichung:
\begin{align}
	\Delta\mathbf{E} - \frac{4\pi\sigma\mu}{c^2} \frac{\partial \mathbf{E}}{\partial t} = 0
	\label{eq:maxwellgleichung}
\end{align}
Wir drücken das elektrische Feld $\mathbf{E}$ durch das ohm'sche Gesetz $\mathbf{j} = \sigma \mathbf{E}$ aus und separieren die Zeitabhängigkeit der Stromdichte \cite{healdmarion} indem wir $\mathbf{j} = \mathbf{j}_0 e^{i \omega t}$ setzen.
\begin{align}
	\Delta\mathbf{j}_0 + k^2 \mathbf{j}_0 = 0
	\label{eq:maxwellgleichungJ}
\end{align}
mit $k^2 = - i \frac{4 \pi \sigma \mu \omega}{c^2}$. Gehen wir in Zylinderkoordinaten über und nutzen Symmetrie aus, erhalten wir:
\begin{align}
	\frac{\mathrm{d}^2 j_0}{\mathrm{d}\rho^2} + \frac{1}{\rho}\frac{\mathrm{d}j_0}{\mathrm{d}\rho} + k^2 j_0 = 0
	\label{eq:besseldgl} 
\end{align}
Multiplikation dieser Gleichung mit $\rho^2$ und der Substitution $\xi = k \rho$ liefert die Bessel'sche Differentialgleichung:
\begin{align}
  \xi^2 \frac{\mathrm{d}^2j_0}{\mathrm{d}\xi^2} + \xi \frac{\mathrm{d}j_0}{\mathrm{d}\xi} + \xi^2 j_0 = 0
\end{align}
Als allgemeine Lösung erhält man Besselfunktionen erster Art und nullter Ordnung, mit der Integrationskonstante $A$:
\begin{align}
	j_0(\rho)=A J_0(k \rho) = A(\mathrm{ber}_0(\kappa \rho)+i\mathrm{bei}_0(\kappa \rho))
\end{align}
wobei für $\kappa$, wie auf dem Blatt definiert, $\kappa = 2 \sqrt{\pi \sigma \mu \omega}/c$ gilt.
Die Konstante $A$ kann aus dem Strom, der im Leiter fließt, bestimmt werden \cite{kazimierczuk}, woraus direkt
\begin{align}
	j_0(\rho) = \frac{I_0 \kappa}{2\pi\rho_0}\frac{\mathrm{ber}_0(\kappa\rho)+i \mathrm{bei}_0(\kappa\rho)}{\mathrm{bei}'_0(\kappa\rho_0)-i\mathrm{ber}'_0(\kappa\rho_0)}
\end{align}
folgt. Mit steigender Frequenz $\omega$ geschieht der Ladungstransport überwiegend am Rand des Leiter. Dieser Effekt wird als \emph{Skin-Effekt} bezeichnet, der hier nach erfolgreicher Implementierung der Kelvin-Funktionen an einigen Werten beispielhaft numerisch überprüft werden soll. Insbesondere ist $j_0$ eine komplexe Zahl, da sie nach dem Separationsansatz sowohl Amplituden wie auch Information über die Phase enthält.


\subsection{Mathematischer Hintergrund}

\subsubsection{Bessel-Funktionen erster Art}

Man kann die Bessel-Funktionen erster Art durch ihre Potenzreihenentwicklung im Ursprung definieren:
\begin{align}
	J_\nu(z) = \sum^{\infty}_{m=0} \frac{\left( -1 \right)^m}{m! \, \Gamma(m + \nu + 1)} \left(\frac{z}{2}\right)^{2m+\nu}
\end{align}
Dabei ist $\nu \in \mathbb{R}$. Uns interessiert die Besselfunktion 0-ter Ordnung $\nu = 0$.
Mit der Identität $\Gamma(n+1) = n!$ für $n \in \mathbb{N}$ ergibt sich:
\begin{align}
	\label{eq:bessel0}
	J_0(z) = \sum^{\infty}_{m=0} \frac{\left( -1 \right)^m}{\left( m! \right)^2} \left( \frac{z}{2} \right)^{2m}
\end{align}

\subsubsection{Kelvin-Funktionen}

Die Kelvin-Funktionen $\mathrm{ber}_\nu(x)$ und $\mathrm{bei}_\nu(x)$ sind definiert als der Real-
beziehungsweise Imaginärteil von $J_\nu(x \sqrt{-i} )$. Mit \ref{eq:bessel0}
folgt für den Fall $\nu = 0$ umgehend die Reihendarstellung der Kelvinfunktionen:
\begin{align}
	\mathrm{ber}_0(x) &= 1 + \sum^{\infty}_{k=1} \frac{\left( -1 \right)^k}{\left( \left(2k\right)! \right)^2} \left( \frac{x}{2} \right)^{4k} \label{eq:berpotenzreihe}\\
	\mathrm{bei}_0(x) &= \sum^{\infty}_{k=0} \frac{\left( -1 \right)^k}{\left( \left(2k+1\right)! \right)^2} \left( \frac{x}{2} \right)^{4k+2}\label{eq:beipotenzreihe}
\end{align}

Für die Ableitung muss (\ref{eq:berpotenzreihe}) und (\ref{eq:beipotenzreihe}) nur gliedweise abgeleitet werden:
\begin{align}
	\mathrm{ber}'_0(x) &= \sum^{\infty}_{k=1} \frac{\left( -1 \right)^k}{\left( 2k-1 \right)! \, \left( 2k \right)!} \left( \frac{x}{2} \right)^{4k-1}\\
	\mathrm{bei}'_0(x) &= \sum^{\infty}_{k=0} \frac{\left( -1 \right)^k}{\left( 2k \right)! \, \left( 2k+1 \right)!} \left( \frac{x}{2} \right)^{4k+1}
\end{align}

In asymptotischer Näherung \cite{abramowitzstegun} ergibt sich mit der Definition $\alpha = \frac{x}{\sqrt{2}}-\frac{\pi}{8}$ und $C(x) = e^{\frac{x}{\sqrt{2}}}/\sqrt{2 \pi x} $:
\begin{align}
	\mathrm{ber}_0(x) &\approx C(x) \left[f_0(x) \cos(\alpha) + g_0(x) \sin(\alpha) \right]\label{eq:berpotenzreiheasym}\\
	\mathrm{bei}_0(x) &\approx C(x) \left[f_0(x) \sin(\alpha) - g_0(x) \cos(\alpha) \right]\label{eq:beipotenzreiheasym}
\end{align}

Mit den Hilfsfunktionen:
\begin{align}
	f_0(x) = 1 + \sum^{\infty}_{k=1} \frac{\cos(k \pi / k)}{k! \, (8k)^k} \prod^{k}_{l=1}(2l - 1)^2\\
	g_0(x) = \sum^{\infty}_{k=1} \frac{\sin(k \pi / k)}{k! \, (8k)^k} \prod^{k}_{l=1}(2l - 1)^2
\end{align}

Auch dieses leiten wir wieder ab:
\begin{align}
	\mathrm{ber}'_0(x) &\approx C(x) \left( \frac{1}{\sqrt{2}} - \frac{1}{2x} \right) \left[f_0(x) \cos(\alpha) + g_0(x) \sin(\alpha) \right] \notag\\ & + C(x) \left[ f'_0(x) \cos(\alpha) + g'_0(x) \sin(\alpha) - f_0(x) \sin(\alpha)/\sqrt{2} + g_0(x) \cos(\alpha)/\sqrt{2} \right]\label{eq:dberpotenzreiheasym}\\
	\mathrm{ber}'_0(x) &\approx C(x) \left( \frac{1}{\sqrt{2}} - \frac{1}{2x} \right) \left[f_0(x) \sin(\alpha) - g_0(x) \cos(\alpha) \right] \notag\\ & + C(x) \left[ f'_0(x) \sin(\alpha) - g'_0(x) \cos(\alpha) + f_0(x) \cos(\alpha)/\sqrt{2} + g_0(x) \sin(\alpha)/\sqrt{2} \right] \label{eq:dbeipotenzreiheasym}
\end{align}

Und den Ableitungen der Hilfsfunktionen:
\begin{align}
	f'_0(x) = 1 - \sum^{\infty}_{k=1} \frac{\cos(k \pi / k)}{(k-1)! \, 8^k} \frac{1}{x^{k+1}} \prod^{k}_{l=1}(2l - 1)^2\\
	g'_0(x) = -\sum^{\infty}_{k=1} \frac{\sin(k \pi / k)}{(k-1)! \, 8^k} \frac{1}{x^{k+1}} \prod^{k}_{l=1}(2l - 1)^2
\end{align}

\subsection{Implementierung}

Sei $S_k$ der $k$-te Summand der Reihendarstellung von \emph{ber}. Wir berechnen $S_k$ aus $S_{k-1}$ mit
\begin{align}
	S_k=c\cdot S_{k-1}
\end{align} mit $c=konstant$.
Analog gilt:\newline
\begin{tabular}{c|c|c|c}
\centering
\rule[1ex]{0pt}{2.5ex} c für $\mathrm{ber}$ & c für $\mathrm{bei}$ & c für $\mathrm{ber}'$ & c für $\mathrm{bei}'$ \\ 
\hline 
\rule[2ex]{0pt}{2.5ex} $-\dfrac{1}{(2k-1)^2\cdot(2k)^2}\dfrac{x^4}{16}$ & $-\dfrac{1}{(2k+1)^2\cdot(2k)^2}\dfrac{x^4}{16}$ & $-\dfrac{1}{(2k-2)(2k-1)^2\cdot 2k}\dfrac{x^4}{16}$ & $-\dfrac{1}{(2k+1)(2k-1) \cdot (2k)^2}\dfrac{x^4}{16}$ \\ 
\end{tabular}




\begin{thebibliography}{9}

\bibitem{abramowitzstegun}
Abramowitz, M. \& Stegun, I. A.
\emph{Handbook of Mathematical Functions},
Dover Books (1965)

\bibitem{healdmarion}
Heald, M. \& Marion, J.
\emph{Classical Electromagnetic Radiation},
Brooks Cole (1994)

\bibitem{kazimierczuk}
Kazimierczuk, M.
\emph{High-Frequency Magnetic Components},
Wiley (2009)

\end{thebibliography}

\end{document}
