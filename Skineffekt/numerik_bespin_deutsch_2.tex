\documentclass[10pt,a4paper]{article}
\usepackage[utf8]{inputenc}
\usepackage[german]{babel}
\usepackage{amsmath}
\usepackage{amsfonts}
\usepackage{amssymb}
\usepackage{siunitx}
\usepackage{multirow}
\usepackage[left=2cm,right=2cm,top=2cm,bottom=2cm]{geometry}
\usepackage{wrapfig}
\usepackage{graphicx}
\usepackage{caption}
\usepackage[colorlinks]{hyperref}


\author{Christian Bespin \and Christopher Deutsch}
\title{Übungsblatt 2: Numerische Methoden der Physik}
\begin{document}
\maketitle

\setcounter{section}{1}

\section{Skin-Effekt}

\subsection{Physikalischer Hintergrund}

In einem zylinderförmigen elektrischen Leiter mit Radius $\rho_0$, der hinreichend lang ist und in dem ein Wechselstrom $I(t)=I_0 e^{i \omega t}$ fließt, ist die Stromdichte nicht über den ganzen Leiterquerschnitt konstant. Das im Leiter schwingende elektrische Feld genügt bei verschwindendem Verschiebungsstrom ($\dot D \approx 0$) der elektromagnetischen Wellengleichung:
\begin{align}
	\Delta\mathbf{E} - \frac{4\pi\sigma\mu}{c^2} \frac{\partial \mathbf{E}}{\partial t} = 0
	\label{eq:maxwellgleichung}
\end{align}
Wir drücken das elektrische Feld $\mathbf{E}$ durch das ohm'sche Gesetz $\mathbf{j} = \sigma \mathbf{E}$ aus und separieren die Zeitabhängigkeit der Stromdichte \cite{healdmarion} indem wir $\mathbf{j} = \mathbf{j}_0 e^{i \omega t}$ setzen.
\begin{align}
	\Delta\mathbf{j}_0 + k^2 \mathbf{j}_0 = 0
	\label{eq:maxwellgleichungJ}
\end{align}
mit $k^2 = - i \frac{4 \pi \sigma \mu \omega}{c^2}$. Gehen wir in Zylinderkoordinaten über, wobei die z-Achse parallel zum Leiter gelegt wird und nutzen Symmetrie aus (keine Winkelabhängigkeit der Stromdichte), erhalten wir:
\begin{align}
	\frac{\mathrm{d}^2 j_0}{\mathrm{d}\rho^2} + \frac{1}{\rho}\frac{\mathrm{d}j_0}{\mathrm{d}\rho} + k^2 j_0 = 0
	\label{eq:besseldgl} 
\end{align}
Multiplikation dieser Gleichung mit $\rho^2$ liefert mit der Substitution $\xi = k \rho$ die Bessel'sche Differentialgleichung:
\begin{align}
  \xi^2 \frac{\mathrm{d}^2j_0}{\mathrm{d}\xi^2} + \xi \frac{\mathrm{d}j_0}{\mathrm{d}\xi} + \xi^2 j_0 = 0
\end{align}
Als allgemeine Lösung erhält man Besselfunktionen erster Art und nullter Ordnung, mit der Integrationskonstante $A$:
\begin{align}
	j_0(\rho)=A J_0(k \rho) = A(\mathrm{ber}_0(\kappa \rho)+i\mathrm{bei}_0(\kappa \rho))
\end{align}
wobei für $\kappa$, wie auf dem Blatt definiert, $\kappa = 2 \sqrt{\pi \sigma \mu \omega}/c$ gilt.
Die Konstante $A$ kann aus dem Strom, der im Leiter fließt, bestimmt werden \cite{kazimierczuk}, woraus direkt
\begin{align}
	j_0(\rho) = \frac{I_0 \kappa}{2\pi\rho_0}\frac{\mathrm{ber}_0(\kappa\rho)+i \mathrm{bei}_0(\kappa\rho)}{\mathrm{bei}'_0(\kappa\rho_0)-i\mathrm{ber}'_0(\kappa\rho_0)}
\end{align}
folgt. Mit steigender Frequenz $\omega$ geschieht der Ladungstransport überwiegend am Rand des Leiter. HIER:(kleine Erklärung mit Wirbelströmen) Dieser Effekt wird als \emph{Skin-Effekt} bezeichnet. Insbesondere ist $j_0$ eine komplexe Zahl, da sie nach dem Separationsansatz sowohl Amplituden- wie auch Phaseinformation enthält.


\subsection{Mathematischer Hintergrund}

\subsubsection{Bessel-Funktionen erster Art}

Man kann die Bessel-Funktionen erster Art durch ihre Potenzreihenentwicklung im Ursprung definieren:
\begin{align}
	J_\nu(z) = \sum^{\infty}_{m=0} \frac{\left( -1 \right)^m}{m! \, \Gamma(m + \nu + 1)} \left(\frac{z}{2}\right)^{2m+\nu}
\end{align}
Dabei ist $\nu \in \mathbb{R}$. Uns interessiert die Besselfunktion 0-ter Ordnung $\nu = 0$.
Mit der Identität $\Gamma(n+1) = n!$ für $n \in \mathbb{N}$ ergibt sich:
\begin{align}
	\label{eq:bessel0}
	J_0(z) = \sum^{\infty}_{m=0} \frac{\left( -1 \right)^m}{\left( m! \right)^2} \left( \frac{z}{2} \right)^{2m}
\end{align}

\subsubsection{Kelvin-Funktionen}
\label{sssec:kelvin-funktionen}

Die Kelvin-Funktionen $\mathrm{ber}_\nu(x)$ und $\mathrm{bei}_\nu(x)$ sind definiert als der Real-
beziehungsweise Imaginärteil von $J_\nu(x \sqrt{-i} )$. Mit (\ref{eq:bessel0})
folgt für den Fall $\nu = 0$ umgehend die Reihendarstellung der Kelvinfunktionen:
\begin{align}
	\mathrm{ber}_0(x) = 1 + \sum^{\infty}_{k=1} \frac{\left( -1 \right)^k}{\left( \left(2k\right)! \right)^2} \left( \frac{x}{2} \right)^{4k}&&
	\mathrm{bei}_0(x) = \sum^{\infty}_{k=0} \frac{\left( -1 \right)^k}{\left( \left(2k+1\right)! \right)^2} \left( \frac{x}{2} \right)^{4k+2}\label{eq:potenzreihe}
\end{align}
Um die erste Ableitung zu erhalten, leiten wir (\ref{eq:potenzreihe}) gliedweise ab:
\begin{align}
	\mathrm{ber}'_0(x) = \sum^{\infty}_{k=1} \frac{\left( -1 \right)^k}{\left( 2k-1 \right)! \, \left( 2k \right)!} \left( \frac{x}{2} \right)^{4k-1}&&
	\mathrm{bei}'_0(x) = \sum^{\infty}_{k=0} \frac{\left( -1 \right)^k}{\left( 2k \right)! \, \left( 2k+1 \right)!} \left( \frac{x}{2} \right)^{4k+1}\label{eq:dpotenzreihe}
\end{align}
In asymptotischer Näherung \cite{abramowitzstegun} ergibt sich mit den Definitionen $\alpha = \frac{x}{\sqrt{2}}-\frac{\pi}{8}$ und $C(x) = e^{\frac{x}{\sqrt{2}}}/\sqrt{2 \pi x} $:
\begin{align}
	\mathrm{ber}_0(x) &\approx C(x) \left[f_0(x) \cos(\alpha) + g_0(x) \sin(\alpha) \right]\label{eq:berpotenzreiheasym}-\frac{\mathrm{kei}_0(x)}{\pi}\\
	\mathrm{bei}_0(x) &\approx C(x) \left[f_0(x) \sin(\alpha) - g_0(x) \cos(\alpha) \right]\label{eq:beipotenzreiheasym}-\frac{\mathrm{ker}_0(x)}{\pi}
\end{align}
Dabei sind $\mathrm{ker}_0$ und $\mathrm{kei}_0$ die Kelvin-Funktionen basierend auf der modifizierten Besselfunktion zweiter Art und nullter Ordnung.
Außerdem gilt:
\begin{align}
	f_0(x) = 1 + \sum^{\infty}_{k=1} \frac{\cos(k \pi / 4)}{k! \, (8k)^k} \prod^{k}_{l=1}(2l - 1)^2&&
	g_0(x) = \sum^{\infty}_{k=1} \frac{\sin(k \pi / 4)}{k! \, (8k)^k} \prod^{k}_{l=1}(2l - 1)^2\label{eq:f0g0}
\end{align}
Wir vernachlässigen den Beitrag von $\mathrm{ker}_0$ und $\mathrm{kei}_0$ zur asymptotischen Näherung, da gilt:
\begin{align}
	\mathrm{ker}_0, \mathrm{kei}_0 \in \mathcal{O}\left( \sqrt{\frac{\pi}{2 x}}e^{-\frac{x}{\sqrt{2}}} \right)
\end{align}
Nun leiten wir (\ref{eq:berpotenzreiheasym}) und (\ref{eq:beipotenzreiheasym}) ab und erhalten:
\begin{align}
	\mathrm{ber}'_0(x) &\approx C(x) \left( \frac{1}{\sqrt{2}} - \frac{1}{2x} \right) \left[f_0(x) \cos(\alpha) + g_0(x) \sin(\alpha) \right] \notag\\ & + C(x) \left[ f'_0(x) \cos(\alpha) + g'_0(x) \sin(\alpha) - f_0(x) \frac{\sin(\alpha)}{\sqrt{2}} + g_0(x) \frac{\cos(\alpha)}{\sqrt{2}} \right]\label{eq:dberpotenzreiheasym}\\
	\mathrm{ber}'_0(x) &\approx C(x) \left( \frac{1}{\sqrt{2}} - \frac{1}{2x} \right) \left[f_0(x) \sin(\alpha) - g_0(x) \cos(\alpha) \right] \notag\\ & + C(x) \left[ f'_0(x) \sin(\alpha) - g'_0(x) \cos(\alpha) + f_0(x) \frac{\cos(\alpha)}{\sqrt{2}} + g_0(x) \frac{\sin(\alpha)}{\sqrt{2}} \right] \label{eq:dbeipotenzreiheasym}
\end{align}
mit
\begin{align}
	f'_0(x) = - \sum^{\infty}_{k=1} \frac{\cos(k \pi / 4)}{(k-1)! \, 8^k} \frac{1}{x^{k+1}} \prod^{k}_{l=1}(2l - 1)^2&&
	g'_0(x) = -\sum^{\infty}_{k=1} \frac{\sin(k \pi / 4)}{(k-1)! \, 8^k} \frac{1}{x^{k+1}} \prod^{k}_{l=1}(2l - 1)^2\label{eq:d_f0g0}
\end{align}

\subsection{Implementierung}

\subsubsection{Struktur des Programmes}
\begin{itemize}
\item \texttt{ber}, \texttt{bei}: Gln. \ref{eq:potenzreihe}
\item \texttt{d\_ber}, \texttt{d\_bei}: Gln. \ref{eq:dpotenzreihe}
\item \texttt{f0}, \texttt{g0}: Gln. \ref{eq:f0g0}
\item \texttt{d\_f0}, \texttt{d\_g0}: Gln. \ref{eq:d_f0g0}
\end{itemize}
Aufteilung in asymptotische Näherung und Reihenentwicklung erklären.

\subsubsection{rekursive Berechnung der Summanden der Reihen}
Um Reihen effizienter zu berechnen (besonders bei Verwendung von Potenzen und Fakultäten), kann man oft den $k$-ten Summanden $S_k$ aus $S_{k-1}$ berechnen. Diesen Fakt nutzen wir aus, um unsere Reihenberechnung effizienter zu gestalten. Allgemein nimmt dies die Form: $S_k = c(x, k)\cdot S_{k-1}$ an.
In Tabelle \ref{tab:rekursionsfaktoren} geben wir die, aus den Gleichungen aus Abschnitt \ref{sssec:kelvin-funktionen}, berechneten Faktoren an.
\begin{wraptable}{R}[1pt]{0.5\textwidth}
\begin{tabular}{c|c}
Funktion & $c$ \\\hline
\rule[-3.5ex]{0pt}{8ex} ber & $-\dfrac{1}{(2k-1)^2 (2k)^2}\dfrac{x^4}{16}$ \\ \hline
\rule[-3.5ex]{0pt}{8ex} bei & $-\dfrac{1}{(2k)^2 (2k+1)^2}\dfrac{x^4}{16}$ \\ \hline
\rule[-3.5ex]{0pt}{8ex} ber' & $-\dfrac{1}{(2k-2) (2k-1)^2 (2k)}\dfrac{x^4}{16}$ \\ \hline
\rule[-3.5ex]{0pt}{8ex} bei' & $-\dfrac{1}{(2k-1) (2k)^2 (2k+1)}\dfrac{x^4}{16}$
\end{tabular}
\caption{Ausgabe unseres Programmes bei vorgegebener Genauigkeit $\epsilon$}
\label{tab:rekursionsfaktoren}
\end{wraptable}

TODO: Das geht bei $f_0$ und $g_0$ sowie den Ableitungen nicht!

\subsubsection{Abbruchbedingungen}

\subsection{physikalische Ergebnisse}
Verwendeter spezifischer Widerstand $\rho_{Cu} = \num{1.678e-8} \si{\ohm\metre}$ \cite{crchandbook}.
Im Gaußschen Maßsystem: $\sigma_{Cu} = \num{5.356e17} \si{\per\second}$

\begin{thebibliography}{9}

\bibitem{abramowitzstegun}
Abramowitz, M. \& Stegun, I. A.
\emph{Handbook of Mathematical Functions},
Dover Books (1965)

\bibitem{healdmarion}
Heald, M. \& Marion, J.
\emph{Classical Electromagnetic Radiation},
Brooks Cole (1994)

\bibitem{kazimierczuk}
Kazimierczuk, M.
\emph{High-Frequency Magnetic Components},
Wiley (2009)

\bibitem{crchandbook}
David R. Lide (ed),
\emph{CRC Handbook of Chemistry and Physics},
84th Edition. CRC Press. Boca Raton, Florida, 2003;
Section 12, Properties of Solids; Electrical Resistivity of Pure Metals

\end{thebibliography}

\end{document}
